\documentclass[12pt]{article}
\usepackage[catalan]{babel}
\usepackage[utf8]{inputenc}
\usepackage{subcaption}
\usepackage[obeyspaces]{url}
\usepackage[colorinlistoftodos]{todonotes}
\usepackage[]{mcode}
\usepackage{fancyvrb}
\usepackage{eurosym} %S�mbol del euro
\usepackage[obeyspaces]{url} %PATH
\usepackage{wrapfig} %Imatges WRAP (mateixa linia)
\usepackage[toc,page]{appendix}
\usepackage{algorithmic}
\usepackage{todonotes}
\usepackage{appendix}
\usepackage{scrextend}
\usepackage{enumitem}
\usepackage{dirtytalk}
\usepackage{multicol}
\usepackage{listingsutf8}
\usepackage{dirtytalk}
\usepackage[utf8]{inputenc}

\UseRawInputEncoding


\setlength{\parindent}{0cm} \setlength{\oddsidemargin}{-0.5cm} \setlength{\evensidemargin}{-0.5cm}
\setlength{\textwidth}{17cm} \setlength{\textheight}{23cm} \setlength{\topmargin}{-1cm} \addtolength{\parskip}{2ex}

\definecolor{backcolour}{rgb}{0.95,0.95,0.92}
\lstdefinelanguage{json}{
	backgroundcolor=\color{backcolour},   
	breaklines=true, 
    string=[s]{"}{"},
    stringstyle=\color{blue},
    comment=[l]{:},
    commentstyle=\color{black}
}


\begin{document}
\begin{titlepage}
		\centering
		\includegraphics[width=0.5\textwidth]{imatges/logo3.png}\par\vspace{1cm}
		{\huge\bfseries Posicionament de restaurants\par}
		\vspace{0.3cm}
		{\scshape\Large Processament de Dades Espaials\par}
		\vspace{0.2cm}
		{\scshape\Large M�ster en enginyeria inform�tica\par}
		\vspace{1.5cm}
		{\Large\itshape Oscar Galera Alfaro i Meriem Abjil Bajja\par}
		\vfill
		{\large \today\par}
\end{titlepage}
\tableofcontents

\clearpage

\listoffigures

\clearpage

\section{Introducci�}
Els imaginadors de Disney han descobert com mantenir el seu univers feli� el m�s petit possible: mitjan�ant l'�s de polseres RFID fredes per fer un seguiment de les identitats, els moviments i l'estat financer dels usuaris.

El MyMagic + "sistema de gesti� de vacances" pot fer un seguiment dels convidats a mesura que es mouen per Walt Disney World i analitzar els seus h�bits de compra.

A partir de la informaci� obtinguda del MyMagic + "sistema de gesti� de vacances", es pot saber quines zones es passeja la majoria de gent.

Degut a l'�xit generat del Walt Disney World, el propietari de Disney Land Paris ha decidit obrir un nou parc tem�tic similar a Disney Land Par�s en el territori Giron�. Ens han encarregat indicar el posicionament dels restaurants amb l�objectiu de maximitzar els beneficis generats per aquests. Per ubicar aquests restaurants, es realitzar� un estudi de la popularitat de les atraccions que hi ha a DLP a fi de determinar les zones m�s visitades i d�aquesta manera, fer la distribuci� dels restaurants m�s cars en aquestes zones per maximitzar els beneficis del parc.

\clearpage
\section{Objectiu}
L'objectiu del projecte �s posar el m�nim de c�meres possibles per cobrir les regions on es troba la majoria de gent. Necessitem que unes determinades regions estiguin cobertes sempre per una o m�s c�meres. Les regions es determinaran a partir del d'un an�lisis de traject�ries (movement patterns) de les dades obtingudes amb el MyMagic + "sistema de gesti� de vacances".

\clearpage
\section{Formalitzaci�}
Bla bla bla
\subsection{Glossari}
Bla bla bla
\subsection{Definici� del problema}
El problema que volem resoldre �s un problema de cobertura 


\subsection{Pre-processament de dades}
Bla bla bla
\subsection{Algoritme}

\clearpage
\section{Articles}

\clearpage
\section{Treball futur}

\clearpage
\section{Conclusions}

\clearpage
\section{Bibliografia}
\end{document}
